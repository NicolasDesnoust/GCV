% !TEX options=--enable-write18
\documentclass[12pt]{article}
\usepackage[french]{babel}
\usepackage{listings}
\usepackage{geometry}
\usepackage{graphicx}
\usepackage{tikz}
\usepackage{adjustbox}
\usepackage[T1]{fontenc}
\usepackage[utf8]{inputenc}
\usepackage{amsfonts}
\usepackage{amsmath}
\usepackage{amssymb}
\usepackage{xcolor, etoolbox}
\usepackage{color, array}
\usepackage{colortbl}
\usepackage[scaled=0.92]{helvet}
\usepackage{enumitem}

\definecolor{rougeBordeaux}{RGB}{87, 36, 40}

%https://pixabay.com/fr/illustrations/carri%C3%A8re-curriculum-vitae-embauche-3449422/
\newcommand{\setbackground}[1] {
	\tikz[remember picture, overlay] \node[opacity=1, inner sep=0pt] at (current page.center) {
		\includegraphics[width=\paperwidth, height=\paperheight]{#1}
	};
}

\newcommand{\important}[1] {\textbf{\textcolor{rougeBordeaux}{#1}}}

\usetikzlibrary{tikzmark,fit}
\geometry{
	a4paper,
	total={170mm,257mm},
	left=20mm,
	top=20mm,
}

\renewcommand{\arraystretch}{1.8} % redéfini la hauteur des lignes d'un tableau
\renewcommand{\familydefault}{\sfdefault}
\renewcommand{\contentsname}{Sommaire} % Renomme Table des matières en Sommaire